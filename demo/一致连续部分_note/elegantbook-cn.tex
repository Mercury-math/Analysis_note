\documentclass[lang=cn,newtx,10pt,scheme=chinese]{elegantbook}

\title{数学分析学习笔记}
\subtitle{复习整理笔记}

\author{阮炜挺}
\institute{宁波大学数学与统计学院}
\date{}

\extrainfo{Given yourself an epsilon of room!}

\setcounter{tocdepth}{3}

\cover{cover.jpg}

% 本文档命令
\usepackage{array}
\newcommand{\ccr}[1]{\makecell{{\color{#1}\rule{1cm}{1cm}}}}

% 修改标题页的橙色带
\definecolor{customcolor}{RGB}{32,178,170}
\colorlet{coverlinecolor}{customcolor}
\usepackage{cprotect}

\addbibresource[location=local]{reference.bib} % 参考文献,不要删除

\begin{document}

\maketitle
\frontmatter

\tableofcontents

\mainmatter

\chapter{一致连续习题}
\begin{example}
    设 $f(x)=\frac{x+2}{x+1}\sin\frac1x,a>0$ 为任一正常数,试证:$f(x)$在$(0,a)$内非一致连续,在$[a,+\infty)$上一致连续.
\end{example}

\begin{example}
    $\text{证明}:f(x)=\frac{1}{x}\text{在}(0,1),g(x)=x^2\text{在}(1,+\infty)\text{内非一致连续}.$
\end{example}

\begin{example}
    证明:$f(x)$在区间$I$上一致连续的充要条件是:对$I$ 上任意两数列
$\left\{x_n\right\},\left\{x_n^{\prime}\right\}$,只要$x_n-x_n^{\prime}\to0$,就有$$f(x_n)-f(x_n^{\prime})\to0(n\to\infty).$$
\end{example}

\begin{example}
    设 $I$ 为有限区间,$f(x)$ 在 $I$ 上有定义,试证:$f(x)$ 在 $I$ 上一致连续的充要条件是 $f$ 把 Cauchy 序列映射为 Cauchy 序列(即当 $\{x_n\}$ 为 Cauchy 序列时,$\{f(x_n)\}$ 亦为 Cauchy 序列)
\end{example}

\begin{example}
    设 $z = g(y)$ 于 $J$, $y = f(x)$ 于 $I$ 都是一致连续的,且 $f(I) \subset J$。试证 $z = g(f(x))$ 在 $I$ 上一致连续。
\end{example}

\begin{example}
    设 $f(x)$ 在有限开区间 $(a, b)$ 内连续,试证 $f(x)$ 在 $(a, b)$ 内一致连续的充要条件是极限 $\lim\limits_{x \to a^+} f(x)$ 及 $\lim\limits_{x \to b^-} f(x)$ 存在(有限)。
\end{example}

\begin{example}
    证明:若 $f(x)$ 在 $[a, +\infty)$ 上连续,$\lim\limits_{x \to +\infty} f(x) = A$(有限),则 $f(x)$ 在 $[a, +\infty)$ 上一致连续。
\end{example}

\begin{example}
    设 $f(x)$ 在 $[a, +\infty)$ 上一致连续,$\varphi(x)$ 在 $[a, +\infty)$ 上连续,
\[
\lim_{x \to +\infty} [f(x) - \varphi(x)] = 0.
\]
证明:$\varphi(x)$ 在 $[a, +\infty)$ 上一致连续。
\end{example}

\begin{example}
设 $f(x)$ 在 $(-\infty, +\infty)$ 上一致连续,则存在非负实数 $a$ 与 $b$,使对一切 $x \in (-\infty, +\infty)$,都有
\[
|f(x)| \leq a |x| + b.
\]
试证明之。
\end{example}

\begin{example}
    设函数 $f(x)$ 在 $[0, +\infty)$ 上一致连续,且 $\forall x > 0$ 有 $\lim\limits_{n \to \infty} f(x+n)=0$ ($n$ 为正整数)。试证 $\lim\limits_{x \to +\infty} f(x)=0$。
\end{example}


\begin{example}
    若 $f(x)$ 在区间 $I$ 上有定义,则 $f(x)$ 在 $I$ 上一致连续的充要条件是
\[
\lim_{\delta \to 0^+} \omega_f(\delta) = 0.
\]
\end{example}

\begin{example}
    设 $E$ 为实轴 $\mathbb{R}$ 上的一个集合,$E_1 \subset E$ 为 $E$ 的稠密子集(即 $\forall x \in E, \exists x_n \in E_1 (n=1,2,\cdots)$,使得 $x_n \to x$(当 $n \to \infty$ 时))。若 $f_1(x)$ 是 $E_1$ 上的一致连续函数(即:$\forall \varepsilon >0, \exists \delta >0$,当 $x',x'' \in E_1, |x'-x''| < \delta$ 时,有 $|f_1(x') - f_1(x'')| < \varepsilon$),则在 $E$ 上有唯一函数 $f(x)$ 使得
    \begin{enumerate}
        \item $f(x) = f_1(x)$(当 $x \in E_1$ 时);
        \item $f(x)$ 在 $E$ 上连续。特别当 $E$ 为有界集合时,$f(x)$ 在 $E$ 上一致连续。
    \end{enumerate}
\end{example}

\section*{习题}
\begin{problem}
    设 \( f \) 是区间 \( I \) 上的实函数,试证如下三条件有逻辑关系:1) $\Rightarrow$ 2) $\Rightarrow$ 3)。
    \begin{enumerate}[label=\arabic*), leftmargin=*, nosep]
        \item \( f \) 在 \( I \) 上可导且导函数有界,即:\(\exists M > 0\) 使得 \( |f'(x)| \leq M \) (\(\forall x \in I\));
        
        \item \( f \) 在 \( I \) 上满足 Lipschitz 条件,即:\(\exists L > 0\) 使得 \( |f(x') - f(x'')| \leq L |x' - x''| \) (\(\forall x', x'' \in I\));
        
        \item \( f \) 在 \( I \) 上一致连续。
    \end{enumerate}
\end{problem}

\begin{problem}
    设 $f( x) $在 区 间 $I$ 上有定义.为了检验$f$在$I$ 上是否一致连续,今设计如下的实验:取一根内空直径为$\varepsilon$的圆形直管( $\varepsilon>0)$,截取长度为$\delta$的一段( $\delta>0)$,将直管中轴与$x$轴平行放好, 然后让$y=f(x)$的曲线平移从管内穿过. 若不论$\varepsilon>0$多么小,只要事先将直管长度$\delta>0$取定足够短,曲线就能平移穿过此管,整个穿越过程 $\delta$ 无须改变,那么$f$就在 $I$ 上一致连续;否则就是非一致连续,问这种理解正确吗?(注 一致性主要体现在“整个穿越过程δ无须改变”上!)
\end{problem}


\begin{problem}
    函数$f(x)$在$[a,b]$上一致连续,又在$[b,c]$上一致连续$,a<b<c.$ 用定义证明$:f(x)$
在$[a,c]$上一致连续
\end{problem}

\begin{problem}
    设$f(x)$在$[0,+\infty)$上满足 Lipschitz 条件,证明$f(x^\alpha)(0<\alpha<1$ 为常数)在$[0,+\infty)$上一致连续。
\end{problem}

\begin{problem}
    证明:$y=\sin\sqrt{x}$在$(0,+\infty)$上一致连续.
\end{problem}

\begin{problem}
    请回答:函数 $f(x)=\sin x^2$在$(-\infty,+\infty)$上是否一致连续?说明理由
\end{problem}

\begin{problem}
    用不等式叙述$f(x)$在$(a,b)$内不一致连续
\end{problem}

\begin{problem}
    证明:$g(x)=\sin\frac{1}{x}$在$(0,1)$内不一致连续.
\end{problem}

\begin{problem}
    证明:函数$f(x)=\frac{|\sin x|}{x}$在每个区间$J_1=\{x|-1<x<0\},J_2=\{x|0<x<1\}$内一致连续,但在$J_1\cup J_2=\{x\mid0<\mid x\mid<1\}$内非一致连续.
\end{problem}

\begin{problem}
    证明:周期函数只要连续必定一致连续
\end{problem}

\begin{problem}
    \begin{enumerate}
        \item 证明:在区间$I$上一致连续的两函数的和与差仍在$I$上一致连续;
        \item 设$f(x),g(x)$是区间$I$上有界且一致连续的函数,求证:$f(x)g(x)$在$I$上一致连续.
    \end{enumerate}
\end{problem}

\begin{problem}
证明:若$(-\infty,+\infty)$上的连续函数 $y=f(x)$有极限
$$\lim_{x\to+\infty}f(\:x\:)\:=A\:,\:\lim_{x\to-\infty}f(\:x\:)\:=B\:,$$
则 $y=f(x)$在$(-\infty,+\infty)$上一致连续.
\end{problem}

\begin{problem}
    设单调有界函数$f$在区间$I(I=(a,b)$或$I=[a,+\infty))$上连续,求证$f$在$l$上一致连续.
\end{problem}

\begin{problem}
    证明:在有限开区间上一致连续的两函数之积仍一致连续.问商的情况怎样?无穷区间上关于积的结论是否还成立?证明之.
\end{problem}

\begin{problem}
    求证:$f(x)=\frac{x^{314}}{\mathrm{e}^x}$在$[0,+\infty)$上一致连续.
\end{problem}

\begin{problem}
    设实函数$f(x)$在$[0,+\infty)$上连续,在$(0,+\infty)$内处处可导,且$\lim \limits_{x\to +\infty} |f^{\prime}(x)|=A($有限或$+\infty$).证明:当且仅当$A$有限时,$f$在$[0,+\infty)$上一致连续.
\end{problem}
\end{document}